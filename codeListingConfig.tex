%%%%%%%%%%%%%%%%%%%%%%%%%%%%%%%%%%%%%%%%%%%%%
%%%%%%%%%%%%%%%%%%%%%%%%%%%%%%%%%%%%%%%%%%%%%
%%%%%%%%%%%%%%%%%%%%%%%%%%%%%%%%%%%%%%%%%%%%%
%%%%%%%%%%%%%%%%%%%%%%%%%%%%%%%%%%%%%%%%%%%%%
%%%%%%%%%%%%%%%%%%%%%%%%%%%%%%%%%%%%%%%%%%%%%
%%%%%%%%%%%%%%%%%%%%%%%%%%%%%%%%%%%%%%%%%%%%%
%%%%%%%%%%%%%%%%%%%%%%%%%%%%%%%%%%%%%%%%%%%%%
%%%%%%%%%%%%%%%%%%%%%%%%%%%%%%%%%%%%%%%%%%%%%
%%%%%%%%% CODE APPEARANCE CONFIG %%%%%%%%%%%%
%%%%%%%%%%%%%%%%%%%%%%%%%%%%%%%%%%%%%%%%%%%%%
%%%%%%%%%%%%%%%%%%%%%%%%%%nicolas igor %%%%%%

%%%%%%%%%%%%%%%%% README %%%%%%%%%%%%%%%%%%%%
% This is a config file to set:
%   1. Style for listing package (could be use on any language)
%   2. Julia syntax highlighting definition
%   3. Command and environment to insert julia, python, and matlab code in your document easily.
%
%       a. new commands: \juliaFile, \pythonFile, and \matlabFile
%           All three are used like \juliaFile[folder_name]{file_name}
%           folder_name is optional, and do it with the slash, e.g. "myFolder/". When no folder is provided, it is assumed that the code is in the same directory as your document.
%           For file_name, do it without extension (it is assumed).
%       Example:
%           \juliaFile[codes/]{kMeans}
%           \juliaFile{randomForest}
%
%       b. new environments: if you want to write your code directly in your document. juliaText, pythonText, and matlabText are provided. Example:
%           \begin{juliaText}{Write your caption here.}
%           Write your code here
%           \end{juliaText}
%%%%%%%%%%%%%%%%%%%%%%%%%%%%%%%%%%%%%%%%%%%%



% Better font encoding, and customized fixed-width font
\usepackage[T1]{fontenc}
\usepackage{inconsolata}

% Colors Definition for Listing Highlighting 
\usepackage{color}
\definecolor{codeRed}{rgb}{0.5,0,0}
\definecolor{codeGreen}{rgb}{0,0.6,0}
\definecolor{codeBlue}{rgb}{0,0,0.5}
\definecolor{codeGray}{rgb}{0.5,0.5,0.5}
\definecolor{codeMauve}{rgb}{0.58,0,0.82}

%%%%%%%%%%%%%%%%%%%%%%%%%%%%%%%%%%%%%%%%%%%%%
%%%%%%%%% LISTING FUN STARTS HERE!!! %%%%%%%%
%%%%%%%%%%%%%%%%%%%%%%%%%%%%%%%%%%%%%%%%%%%%%
\usepackage{listings} % For displaying code

% General Appearance Settings in an Isolated Style
\lstdefinestyle{myStyle}{ %
  basicstyle=\small\ttfamily,      % the size of the fonts that are used for the code
  basewidth=0.5em,                 % sets size of reserved space on each column
  breakatwhitespace=true,          % sets if automatic breaks should only happen at whitespace
  breaklines=true,                 % sets automatic line breaking
  commentstyle=\color{codeGreen},  % comment style
  extendedchars=true,              % lets you use non-ASCII characters; for 8-bits encodings only, does not work with UTF-8
  frame=tb,	                       % adds a frame around the code
  keywordstyle=\color{codeBlue},       % keyword style
  numbers=left,                    % where to put the line-numbers; possible values are (none, left, right)
  numbersep=5pt,                   % how far the line-numbers are from the code
  numberstyle=\scriptsize\color{codeGray}, % the style that is used for the line-numbers
  rulecolor=\color{black},         % if not set, the frame-color may be changed on line-breaks within not-black text (e.g. comments (green here))
  showspaces=false,                % show spaces everywhere adding particular underscores; it overrides 'showstringspaces'
  showstringspaces=false,          % underline spaces within strings only
  showtabs=false,                  % show tabs within strings adding particular underscores
  stepnumber=1,                    % the step between two line-numbers. If it's 1, each line will be numbered
  numberfirstline=true, 
  numberblanklines=true,
  stringstyle=\color{codeMauve},     % string literal style
  tabsize=4,	                   % sets default tabsize to 4 spaces
}

% Definition of Julia Highlighting
\lstdefinelanguage{Julia}%
{ language = Matlab,% 
    keywordsprefix=\@,%
    morekeywords={
    exit,whos,edit,load,is,isa,isequal,typeof,tuple,ntuple,uid,hash,finalizer,convert,promote,
    subtype,typemin,typemax,realmin,realmax,sizeof,eps,promote_type,method_exists,applicable,
    invoke,dlopen,dlsym,system,error,throw,assert,new,Inf,Nan,pi,im,begin,while,for,in,return,
    break,continue,macro,quote,let,if,elseif,else,try,catch,end,bitstype,call,do,using,module,
    import,export,importall,baremodule,immutable,local,global,const,Bool,Int,Int8,Int16,Int32,
    Int64,Uint,Uint8,Uint16,Uint32,Uint64,Float32,Float64,Complex64,Complex128,Any,Nothing,nothing,
    function,type,typealias,abstract,case,false,otherwise,switch,true,include
  },%
  sensitive=true, %
  comment=[l]{\#}, % l is for line comment
  morecomment=[s]{\#=}{=\#}, % s is for start and end delimiter
  string=[b]" % defines that strings are enclosed in double quotes
}

% Command to include files of code easier, Julia
\newcommand{\juliaFile}[2][]{\lstinputlisting[caption=Script \emph{#2}., label=code:#2, style=myStyle, language=Julia]{#1#2.jl}}
% Command to include files of code easier, Python
\newcommand{\pythonFile}[2][]{\lstinputlisting[caption=Script \emph{#2}., label=code:#2, style=myStyle, language=Python]{#1#2.py}}
% Command to include files of code easier, Matlab
\newcommand{\matlabFile}[2][]{\lstinputlisting[caption=Script \emph{#2}., label=code:#2, style=myStyle, language=Matlab]{#1#2.m}}

% Environment to write by hand code easier, Julia
\lstnewenvironment{juliaText}[1]{\lstset{caption=#1, style=myStyle, language=Julia, firstnumber=1}}{}
% Environment to write by hand code easier, Python
\lstnewenvironment{pythonText}[1]{\lstset{caption=#1, style=myStyle, language=Python, firstnumber=1}}{}
% Environment to write by hand code easier, Matlab
\lstnewenvironment{matlabText}[1]{\lstset{caption=#1, style=myStyle, language=Matlab, firstnumber=1}}{}

%%%%%%%%%%%%%%%%%%%%%%%%%%%%%%%%%%%%%%%%%%%%%
%%%%%%%%%%% LISTING FUN ENDED  %%%%%%%%%%%%%%
%%%%%%%%%%%%%%%%%%%%%%%%%%%%%%%%%%%%%%%%%%%%%